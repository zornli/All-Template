%!TEX TS-program = xelatex
%!TEX encoding = UTF-8 Unicode

\documentclass[12pt,a4paper,twoside]{article}
\usepackage[top=1.7cm,bottom=1.2cm,left=2.8cm,right=2.3cm,includehead,includefoot]{geometry}
%\geometry{letterpaper} commt

\usepackage{fontspec,xltxtra,xunicode}
\defaultfontfeatures{Mapping=tex-text}

\usepackage{fancyhdr}         
\usepackage{indentfirst}           

\usepackage{amsmath}               
\usepackage{amssymb}                   % AMSLaTeX\usepackage{amsfonts}                  % AMSLaTeX
\usepackage[amsmath,thmmarks]{ntheorem}% AMSLaTeX

\usepackage{graphicx}                  %
\usepackage{floatflt}
\usepackage{wrapfig}
\usepackage{picinpar} 

\usepackage{mathrsfs}               %\mathcal or \mathfrak 

\usepackage{color}                 
\definecolor{Gold}{rgb}{1.00, 0.84, 0.00}    
\definecolor{Orange}{rgb}{1.00, 0.65, 0.00}

\usepackage[colorlinks]{hyperref}   
\setlength{\topsep}{-16pt }                        %\setlength{\partopsep}{2pt plus0pt minus2pt}        % 
\setlength{\itemsep}{-16pt }                        % \setlength{\floatsep}{10pt plus 3pt minus 2pt}      % 
\setlength{\abovecaptionskip}{2pt plus1pt minus1pt} % 
\setlength{\belowcaptionskip}{3pt plus1pt minus2pt} % 
\setlength{\floatsep}{10pt plus 3pt minus 2pt}
\setlength{\parindent}{2em}                         % 
\setlength{\parskip}{2pt plus1pt minus1pt}          % 
\renewcommand{\baselinestretch}{1.2}                % 



%%=========================================================================%%==================================%
\bibliographystyle{plain}         
%=================
\makeatletter                                                                                                 
\def\@cite#1#2{\textsuperscript{[{#1\if@tempswa,#2\fi}]}}
\makeatother
%==============================================================================================================
\vfuzz2pt % Don't report over-full v-boxes if over-edge is small
\hfuzz2pt % Don't report over-full h-boxes if over-edge is small



\setromanfont{Songti SC Regular} %设置中文字体
\XeTeXlinebreaklocale “zh”
\XeTeXlinebreakskip = 0pt plus 1pt minus 0.1pt %文章内中文自动换行

\newfontfamily{\H}{Hei}
\newfontfamily{\K}{Kai}
\newfontfamily{\F}{FangSong}
\newfontfamily{\E}{Arial}

\title{\H 算法设计}
\author{Donald}
%\date{\E\today}
\date{}

\begin{document}

\maketitle


\theoremstyle{plain}
\theoremheaderfont{\setromanfont{Hei}}
\theorembodyfont{\setromanfont{Kai}} \theoremindent0em
\theoremseparator{\hspace{1em}} \theoremnumbering{arabic}
\theoremsymbol{} 
\newtheorem{theorem}{\hspace{2em}定理}
\newtheorem{definition}{\hspace{2em}定义}
\newtheorem{lemma}{\hspace{2em}引理}
\newtheorem{corollary}{\hspace{2em}推论}

\newcommand{\ET}{\mathcal{T}}     
\newcommand{\ES}{\mathcal{S}}
\newcommand{\EN}{\mathcal{N}}
\newcommand{\EI}{\mathcal{I}}
\newcommand{\EF}{\mathcal{F}}
\newcommand{\HF}{\mathscr{F}} 

\renewcommand\refname{\small \H 参考文献}         
\renewcommand\tablename{\small \H 表}

\pagestyle{fancy}
\fancyhead{}                                   
\fancyhead[CE]{ }                              
\fancyhead[CO]{{\small 页眉文字}}   
\fancyhead[RO]{\thepage}          
\fancyhead[LE]{\thepage}           
\fancyfoot[C]{}   

\renewcommand{\headrulewidth}{0.4pt}      
\addtolength{\headsep}{-1em}                
\sloppy    

\begin{quote}\small
{\H\quad 摘~~~要:}
          {\K 中文\LaTeX{}
                 }

 {\H \quad 关键词:} { \LaTeX{}; \TeX{}; $\mathbb{C}$\!\TeX{}}
\end{quote}

\noindent\hspace*{5cm} \hrulefill \hspace*{5cm}\\    

%\vspace{-0.8cm}
             

这个段落中,夹杂着一个{\E word}。
{\K 看看效果}。{ \F 仿宋。}


\begin{definition}[中文]\label{def:dubois} % 
{\begin{eqnarray}
 \begin{split}
 \mu_{\underline{\Re}\mathcal {F}}\left( x\right)
      &= \inf\left\{\max \left[ \mu _{\HF} \left( y \right),1 - \mu_\Re\left(x,y\right) \right]\bigm| y \in U  \right\} \\
 \mu_{\overline{\Re}\mathcal {F}}\left( x\right)
      &= \sup\left\{{\min \left[ \mu _{\HF} \left( y \right),\mu_\Re\left(x,y\right) \right]\bigm| y \in U}\right\}
 \end{split}
\end{eqnarray}\label{eq:dubois1}}
\end{definition}

\begin{definition}[中文定义]

中文$\left( U,\Re \right)$看看$\Re$中文$U$看看 $\forall
\EF\in\HF(U)$以及$\EF$很好 $\left(U,\Re\right)$ 看啊困难
$\underline{\Re}\mathcal {F}$什么$\overline{\Re}\mathcal
{F}$以及$U$继续
 {\begin{eqnarray}
 \begin{split}
 \mu_{\underline{\Re}\mathcal {F}}\left( x\right)
      &= \inf\left\{\max \left[ \mu _{\HF} \left( y \right),1 - \mu_\Re\left(x,y\right) \right]\bigm| y \in U  \right\} \\
 \mu_{\overline{\Re}\mathcal {F}}\left( x\right)
      &= \sup\left\{{\min \left[ \mu _{\HF} \left( y \right),\mu_\Re\left(x,y\right) \right]\bigm| y \in U}\right\}
 \end{split}
\end{eqnarray}\label{eq:dubois1}}

\end{definition}


\clearpage1

\section{\F 第一部分}

文章

\clearpage

看看效果。\cite{Biswas}


\begin{thebibliography}{10}

\scriptsize \addtolength{\itemsep}{-1.0em}    


\bibitem{IntuitionisticFuzzySet} 
Atanassov K.
\newblock Intuitionistic \textit{L}-fuzzy sets.
\newblock {\em Fuzzy Sets and Systems}, 20:87$\sim$96, 1986.






\bibitem{Sankal96RoughnessofFuzzy}
 Banerjee M, Pal S.~K.                                        
\newblock Roughness of fuzzy set.                             %title
\newblock {\em Information Sciences}, 93:235$\sim$246, 1996. 



\bibitem{Biswas}
 Biswas R.
\newblock On rough sets and fuzzy rough sets.
\newblock {\em Bulletin of the Polish Academy of Sciences,Mathematics},
  42:345$\sim$349, 1994.




\end{thebibliography}

%=============================
\vspace{2em}
\begin{center}
 \textbf{\Large      }\\[2em]

 HUANG Zheng-hua \\[0.5em]
 (~{\small \textit{School of Mathematics and Statistics, Wuhan University, Wuhan 430072, Hubei, P.R.China}~})
\end{center}

\vspace{0.5cm}
\noindent\textbf{Abstract}: This is a abstract.

 \noindent\textbf{Key words}:  \LaTeX{}; \TeX{}; $\mathbb{C}$\!\TeX{}


%====================================
\clearpage


\end{document}

